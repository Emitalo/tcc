\chapter{Conclusão} \label{cap:consideracoes_finais}

  As plataformas digitais têm auxiliado no exercício da democracia, aproximando o cidadão da política, principalmente
  através de discussões sobre políticas e projetos. Contudo,
  essas plataformas geram uma polarização das mensagens trocadas e formação de bolhas de opinião, 
  pois não estão estruturadas para apresentar opiniões sem viés.
  
  Nesse sentido, o Instituto Cidade Democrática apresentou a ideia da plataforma ``Empurrando Juntos'' com o objetivo
  de minimizar a polarização dessas discussões ao permitir que todas as opiniões sejam visualizadas por meio de agrupamentos
  dos usuários de acordo com seus pontos de vista. Sendo que estes agrupamentos seriam realizados por meio de técnicas 
  de classificação de dados.

  Neste trabalho foi proposta uma arquitetura modular para essa nova plataforma, da qual o módulo de serviços ou API
  foi implementado e validado com os serviços de gestão de usuários, 
  conversas, comentários, votos e formação de grupos. Este último com a possibilidade da utilização de qualquer método 
  de classificação de dados.
  
  Para a implementação da API primeiramente foram identificados os requisitos funcionais e não funcionais da aplicação
  relacionados ao escopo a ser tratado no trabalho, além
  da definição das tecnologias a serem utilizadas. A escolha do \textit{framework} Django foi considerada apropriada, pois permitiu
  a separação em \textit{apps} provendo um baixo acoplamento na aplicação. 
  
  Em relação ao nível de maturidade da API, foi percebido que ela se encaixa no Nível 2 proposto,
  pois utiliza corretamente os verbos e códigos de resposta HTTP.
  A utilização da especificação JSONAPI motrou-se um primeiro passo para atingir o Nível 3 de maturidade,
  adicionando alguns \textit{hyperlinks} de controle nas respostas da API.
  
  Considerando a arquitetura modularizada do sistema e a API como sendo apenas uma parte do todo, 
  foram conectados os outros dois módulos para a validação 
  da solução. Na comunicação com aplicação cliente foi possível confirmar a escolha adequada da arquitetura dos serviços e 
  constatar o correto funcionamento dos \textit{endpoints}.
  
  Na conexão com o módulo matemático já existente, contudo adaptado para o novo protocolo, foi possível validar a interface de 
  comunicação definida e no conjunto dos módulos, a formação dos grupos. Com isso foi possível verificar e validar o escopo 
  inicial da aplicação de acordo com a proposta do ``Empurrando Juntos'', incluindo a escolha da clusterização 
  como um primeiro método para a realização dos agrupamentos. 
  
  Dessa forma, os resultados obtidos com a validação utilizando apenas uma opção de aplicação de classificação são considerados pertinentes.
  Porém é reconhecida como limitação deste trabalho a ausência de um teste da API com outros algoritmos de classificação implementados em outras aplicações.
  Nesse sentido, um trabalho futuro proposto é a implementação de outros módulos matemáticos com diferentes técnicas
  de classificação dos dados para validação do protocolo apresentado e para estudos de métodos mais apropriados para o contexto.
  
  Tratando da contribuição para o ``Empurrando Juntos'' e do escopo tratado neste trabalho, outra proposta de trabalho futuro é a 
  implementação das demais funcionalidades da plataforma não desenvolvidas neste trabalho,
  como por exemplo, o elemento de gamificação.
  
  Por fim, como a arquitetura proposta foi definida apenas para a utilização de um módulo matemático por vez, em outro trabalho futuro, 
  essa estrutura pode ser expansível para o funcionamento do sistema com mais de um módulo ao mesmo tempo, atribuindo a escolha ao usuário da aplicação.
  