\chapter{Considerações Finais} \label{cap:consideracoes_finais}

Este trabalho apresentou a proposta de uma API para uma plataforma de
participação social e sua adaptação para o contexto de violência contra a mulher.

Nesse primeiro momento, foi realizada a etapa de diagnóstico com o intuito de 
analisar a ideia do ``Empurrando Juntos''. A partir dessa etapa, foi possível entender
o seu atual funcionamento, que consiste apenas pelo módulo de clusterização, e por consequência identificar as necessidades da plataforma.
Além disso, nessa etapa foi realizado um levantamento das aplicações existentes na temática de violência contra a mulher com o intuito de 
entender suas contribuições e lacunas para este contexto.

A partir das informações obtidas na etapa de diagnóstico, foi realizada a segunda etapa do trabalho, na qual as necessidades identificadas
foram convertidas em uma proposta de solução.

Nesse contexto, no final do trabalho é esperado uma API REST que seja capaz de fornecer um serviço de participação com conversas, comentários
e votos que agrupe pessoas a partir de suas opiniões. A criação da API contribuirá para a formação da plataforma ``Empurrando Juntos'' 
através da implementação da parte servidor da arquitetura definida.

No contexto social, o objetivo é que a API possa contribuir no apoio à mulheres vítimas de violência dentro das plataformas existentes, através
da promoção de discussões mais efetivas, da possibilidade de promover políticas públicas e da criação de uma rede de apoio entre mulheres que passam ou passaram por essa situação.
