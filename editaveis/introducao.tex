\chapter[Introdução]{Introdução} \label{cap:introducao}
% \addcontentsline{toc}{chapter}{Introdução}

A democracia digital é resumida por \citeonline{penteadoencontro} como o uso da Internet para consolidação da democracia.
Esse uso tem acarretado em um crescente número de discussões acerca de temas políticos, o que permitiu que interessados na área analisassem 
esse fenômeno e observassem uma polarização das mensagens trocadas nas redes sociais \cite{empurrandojuntos}.

\citeonline{empurrandojuntos} afirmaram que as discussões realizadas, principalmente em redes sociais, 
acabavam refletindo sempre a opinião da maioria e que as pessoas estão sempre presentes em uma bolha de opinião. 
Isto é, os algoritmos dessas plataformas selecionam o conteúdo a ser apresentado de acordo com o comportamento anterior,
no qual foi coletada a opinião deste usuário.
Dessa forma, essa polarização dificulta a explanação das ideias da minoria e restringe a apresentação de novas ideias ou pensamentos diferentes 
para quem usa essas redes sociais. 

Observando esse aspecto, o Instituto Cidade Democrática\footnote{Site do Cidade Democrática: \url{http://www.cidadedemocratica.org.br/}} 
apresenta a ideia da plataforma ``Empurrando Juntos'' cujo objetivo 
é dar voz para a minoria e tornar as discussões mais efetivas para os seus propósitos \cite{empurrandojuntos}.

A ideia é que um usuário entre em um website ou em um aplicativo de celular para criar conversas e 
participar de conversas criadas por outros usuários. Essa participação 
acontece de duas formas: comentando em uma conversa ou votando em um comentário de outro participante. Entende-se por voto
o ato de concordar com o comentário realizado (uma espécie de \textit{like}) ou discordar do comentário. Além disso, é permitido
que o usuário pule aquele comentário, ou seja, não atribua nenhum tipo de voto \cite{empurrandojuntos}. 

Com os votos realizados, a ideia é agrupar pessoas que responderam de maneira parecida, ou seja, concordaram e
discordaram dos mesmos comentários. Com os grupos formados, é possível ver a convergência e divergência de opiniões, 
prover ao usuário uma visão ampliada das opiniões acerca do assunto e promover a interação entre os usuários com 
pensamentos divergentes.

Esse agrupamento é realizado em tempo real, ou seja, conforme as pessoas votam são formados ou modificados os grupos. 
Isto é, não há estabelecimento prévio dos grupos, somente se sabe que os usuários têm dados em comum, que são os votos nas conversas. Essa característica determinou a implementação deste módulo da plataforma utilizando a técnica de clusterização.

Atualmente, o módulo que realiza a clusterização é a única funcionalidade em desenvolvimento na plataforma. Considerando a necessidade do
``Empurrando Juntos'' de criar uma aplicação e aplicativos, uma arquitetura com o uso de uma \textit{Application Programming Interface} (API) foi
definida. API é uma interface que expõe os seus componentes como um serviço, permitindo que outras aplicações interajam com esses 
componentes \cite{wagh2012comparative, understanding_web}. Essa arquitetura de serviços possibilita o compartilhamento dos dados 
armazenados com as aplicações que consomem o serviço. Além disso, 
seu uso permite a convergência de informações e disseminação da sua proposta por meio de seu uso em diversos tipos de sistema.

A ideia do ``Empurrando Juntos'' de incentivar esse tipo de discussão, com a exposição de todas as opiniões obtidas, é essencial para possibilitar mudanças neste âmbito político, como
para debates sobre projetos de leis e até mesmo em outros assuntos de relevância, para promoção da criação de políticas públicas.

Nesse contexto, uma das temáticas que tem se destacado nessas discussões é a de violência contra a mulher. Segundo \citeonline{violence_against_women}, 
o termo violência contra a mulher trata de
diversos atos de abuso e violência baseados no gênero, ou seja, dirigidos à mulheres e meninas ao longo da vida.
De acordo com o estudo da \citeonline{violence_global}, dos diversos tipos de violência existentes, a violência doméstica, ou proveniente do parceiro,
e a violência sexual, proveniente de um indivíduo diferente do parceiro, são as formas de violência que prevalecem.

Esse destaque da temática é impulsionado pelas estatísticas e pela nova geração, que busca defender o direito das mulheres. \citeonline{violence_global}, em sua pesquisa, mostra que 
35\% das mulheres no mundo já vivenciaram 
uma situação de violência física e/ou sexual pelo parceiro e violência sexual por outro indivíduo. No mundo, 38\% dos homicídios de mulheres são cometidos 
pelos parceiros das vítimas.

No Brasil, de acordo com o Sistema de Informações sobre Mortalidade (SIM), entre 1980 e 2013, 106.093 mulheres foram vítimas de homicídio, 
representando em 2013 uma taxa de aproximadamente 13 homicídios femininos
diários \cite{mapa_violencia_2015}. 

Cenários como esses impulsionaram além das discussões promovidas nas redes sociais, a criação de políticas públicas para a redução da violência 
contra as mulheres e de estratégias de apoio às mulheres e de conscientização da população. Além disso, estratégias tecnológicas, 
como aplicativos e sites, têm surgido nesse contexto.

No Brasil, a criação da Lei Maria da Penha, da Lei do Feminicídio e de programas e serviços de apoio à causa, 
como o disque-denúncia Ligue 180, a Casa da Mulher Brasileira e a Unidade Móvel de Atendimento são respostas aos cenários supracitados. 
No contexto de Tecnologia da Informação (TI), a criação de software têm sido apoiada pelo governo e realizada pela própria população.

Um levantamento que realizamos sobre as aplicações existentes no Brasil, demonstra que o apoio proveniente da tecnologia concentra em 
prover uma rede de denúncias para mapear locais de risco ou levar essa denúncia até uma autoridade competente. Além disso, muitas aplicações 
preocupam-se em prover informações sobre leis e conceitos.

Contudo, o cenário de violência contra a mulher no Brasil ainda deve ter atenção do governo e da sociedade. No primeiro semestre de 2016 
foram contabilizados 555.634 atendimentos na central de denúncias 
de violência contra a mulher, de acordo com o levantamento feito pela Secretaria de Políticas para as Mulheres (SPM). 
Aproximadamente 54\% dos atendimentos foram para prestação de informações. De acordo com este levantamento \cite{portal_180}, aproximadamente 13\% dos 
atendimentos, 
são relatos de violência física (51\%), psicológica (31,1\%), moral (6,51\%), patrimonial (1,93\%), sexual (4,30\%), cárcere privado (4,86\%) e 
tráfico de pessoas (0,24\%).

Observando esses aspectos, percebe-se que mesmo com as políticas e aplicações existentes, a promoção de discussões efetivas acerca do 
assunto ainda é abaixo do esperado. Assim, nota-se a importância de promover essas discussões sobre a situação das mulheres a fim de 
melhorá-la em diversos aspectos, seja criando uma rede de apoio entre as mulheres que enfrentam situações semelhantes ou chamando a atenção 
para causas de interesse público. Essa ideia de promover discussões mais efetivas é compatível com o objetivo da plataforma ``Empurrando Juntos''.

Considerando os aspectos supracitados, o objetivo do trabalho é criar uma API para prover o gerenciamento das conversas e formação
dos grupos de pessoas para a plataforma ``Empurrando Juntos'' e avaliá-la, evoluindo-a para o uso em plataformas de apoio à mulheres 
vítimas de violência.


% \section{Etapas do trabalho}

Para realização do objetivo proposto o trabalho foi dividido em cinco etapas, conforme a Figura \ref{fig:etapas_trabalho}. 
Este trabalho de conclusão de curso 1 
teve como escopo as três primeiras etapas.

\begin{figure}[h!]
\centering
\includegraphics[scale=0.6]{figuras/etapas.png}
\caption{Etapas do trabalho}
\label{fig:etapas_trabalho}
\end{figure}

A etapa de ``Diagnóstico'' compreendeu o entendimento do escopo da plataforma ``Empurrando Juntos'' e do cenário das 
aplicações implementadas no contexto de violência contra a mulher. A etapa de ``Definição da Proposta'' foi caracterizada pela definição de escopo, 
da arquitetura da API e da comunicação com o módulo de clusterização. Na terceira etapa, foi feito o planejamento da execução do trabalho com o estabelecimento
de um cronograma com as atividades a serem realizadas nas próximas etapas. 

As duas últimas etapas serão realizadas posteriormente. Na etapa de ``Desenvolvimento''
serão realizadas as iterações de implementação, teste e adaptação da API. Por fim, na etapa de ``Aplicação em um caso'' a API será utilizada em uma 
plataforma de exemplo.

% Levantar sistemas relacionados, Mapear funcionalidades em comum, Definir escopo da API,
% Modelar estrutura da API, Desenvolver a API e Desenvolver um caso de exemplo.

% Passos: Levantar sistemas relacionados, definir escopo da api, definir tecnologia, 
% definir um questionário padrão a ser respondido pelas mulheres para apoio a tomada de decisão, implementar api, definir categorias, adicionar inteligência, definir ações

Nesse contexto, este trabalho, além desta introdução, está organizado em outros cinco capítulos. No Capítulo \ref{cap:sistemas_relacionados} são 
apresentados os sistemas existentes de apoio às mulheres vítimas de violência. 
No Capítulo \ref{cap:api} são apresentados os conceitos e arquiteturas para APIs. O Capítulo \ref{cap:clusterizacao} aborda os conceitos da técnica de 
classificação utilizada no ``Empurrando Juntos''. 
A proposta do trabalho é apresentada no Capítulo \ref{cap:proposta} e por fim são apresentadas as considerações finais no Capítulo 
\ref{cap:consideracoes_finais}.




% Diante desse cenário, é percebido que cada aplicação fornece apoio para a mulher baseado em leis e gera dados importantes para o contexto, 
% porém esses dados ficam apenas no domínio daquela aplicação e assim diminuindo a efetividade das estatísticas geradas. 
% Além disso, as aplicações partem do princípio de que as mulheres sabem que sofreram algum tipo de violência. Outro ponto percebido, 
% demonstra que 
% 
% 
% Observando esses aspectos, nota-se a necessidade de consolidar os dados provenientes das denúncias. 
% Ademais, auxiliar as mulheres a descobrirem se sofreram violência e se apoiarem. 
% Estas necessidades identificadas podem ser apoiadas pela tecnologia, todavia necessitam de 
% um apoio especializado, assim como é feito nas aplicações existentes.
% 
%  

% Atualmente, novas plataformas têm surgido com o intuito de melhorar a efetividade das discussões promovidas na Internet. 
% Uma dessas plataformas é o Empurrando Juntos. A ideia é mudar o fluxo de comunicação que tem se formado nas redes socias, 
% a fim de permitir que todos sejam ouvidos e privilegiar aqueles que se encontram na minoria, 
% os quais perdem a visibilidade das suas mensagens \cite{empurrandojuntos}. 
% Em suma, a plataforma "é um aplicativo de conversa na rede com uma interface minimalista de participação que identifica e 
% exibe grupos de opinião e propostas a partir dos dados de participação" \cite{empurrandojuntos}.
%  
% % O I-DECIDE é um projeto australiano, proveniente dessas estratégias, responsável por disponibilizar para as mulheres um site no qual elas 
% podem avaliar sua relação, suas prioridades e planejar um futuro mais seguro. A ferramenta compreende avaliações de segurança e um processo de 
% planejamento de ação individualizado adaptado às circunstâncias particulares de cada mulher \cite{idecide}.
% 
% 
% 
% 
% das denúncias e permite a criação de uma rede de apoio mais especializado por meio de questionários elaborados por estudiosos da temática.

% \section{Objetivos}

% Considerando os aspectos supracitados, o objetivo do trabalho é criar uma API que consolide as informações das aplicações existentes e forneça informações para estudos que auxilie na descoberta de uma possível situação de violência e na criação de uma rede de apoio à mulher por meio de clusterização.

