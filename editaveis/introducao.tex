\chapter[Introdução]{Introdução} \label{cap:introducao}
% \addcontentsline{toc}{chapter}{Introdução}

Segundo \citeonline{violence_against_women}, o termo violência contra a mulher trata de
diversos atos de abuso e violência baseados no gênero, ou seja, dirigidos à mulheres e meninas ao longo da vida.
De acordo com o estudo da \citeonline{violence_global}, dos diversos tipos de violência existentes, a violência doméstica, ou proveniente do parceiro,
e a violência sexual, proveniente de um indivíduo diferente do parceiro, são as formas de violência que prevalecem.

A pesquisa relatada pela \citeonline{violence_global} mostra que 35\% das mulheres no mundo já vivenciaram uma situação
de violência pelo física e/ou sexual pelo parceiro e violência sexual por outro indivíduo. No mundo, 38\% dos homicídios de mulheres são cometidos pelos parceiros das vítimas.

No Brasil, de acordo com o Sistema de Informações sobre Mortalidade (SIM), entre 1980 e 2013, 106.093 mulheres foram vítimas de homicídio, representando em 2013 uma taxa de aproximadamente 13 homicídios femininos
diários \cite{mapa_violencia_2015}. 
No primeiro semestre de 2016 foram contabilizados 555.634 atendimentos na central de denúncias 
de violência contra a mulher, de acordo com o levantamento feito pela Secretaria de Políticas para as Mulheres (SPM). 
Aproximadamente 54\% dos atendimentos foram para prestação de informações. De acordo com \cite{portal_180}, aproximadamente 13\% dos atendimentos, são relatos de violência física (51\%), psicológica (31,1\%), moral (6,51\%), patrimonial (1,93\%), sexual (4,30\%), cárcere privado (4,86\%) e tráfico de pessoas (0,24\%).

Ao longo dos anos, cenários como esses impulsionaram os governos à criação de políticas públicas (PP) para a redução da violência contra as mulheres e de estratégias de apoio às mulheres e de conscientização da população. Além disso, estratégias tecnológicas, como aplicativos e sites, têm surgido nesse contexto.

% O I-DECIDE é um projeto australiano, proveniente dessas estratégias, responsável por disponibilizar para as mulheres um site no qual elas podem avaliar sua relação, suas prioridades e planejar um futuro mais seguro. A ferramenta compreende avaliações de segurança e um processo de planejamento de ação individualizado adaptado às circunstâncias particulares de cada mulher \cite{idecide}.

No Brasil, a criação da Lei Maria da Penha, da Lei do Feminicídio e de programas e serviços de apoio à causa 
como o Disque-denúncia Ligue 180, a Casa da Mulher Brasileira e a Unidade Móvel de Atendimento são respostas aos cenários supracitados. No contexto de Tecnologia da Informação (TI), a criação de software têm sido apoiada pelo governo e realizada pela própria população.

Um rápido levantamento realizado sobre as aplicações existentes no Brasil, demonstra que o apoio proveniente da tecnologia concentra em prover uma rede de denúncias para mapear locais de risco ou levar essa denúncia até uma autoridade competente. Além disso, muitas aplicações preocupam-se em prover informações sobre leis e conceitos.

Diante desse cenário, é percebido que cada aplicação fornece apoio para a mulher baseado em leis e gera dados importantes para o contexto, porém esses dados ficam apenas no domínio daquela aplicação e assim diminuindo a efetividade das estatísticas geradas. Além disso, as aplicações partem do princípio de que as mulheres sabem que sofreram alguma tipo de violência. 

Observando esses aspectos, percebe-se que mesmo com as políticas e aplicações existentes, a promoção de discussões efetivas acerca do assunto ainda é abaixo do esperado, assim nota-se a importância de promover essas discussões sobre a situação das mulheres a fim de melhorá-la em diversos aspectos, seja criando uma rede de apoio entre as mulheres que enfrentam situações semelhantes ou chamando a atenção para causas de interesse público.

% Observando esses aspectos, nota-se a necessidade de consolidar os dados provenientes das denúncias. Ademais, auxiliar as mulheres a descobrirem se sofreram violência e se apoiarem. Estas necessidades identificadas podem ser apoiadas pela tecnologia, todavia necessitam de 
% um apoio especializado, assim como é feito nas aplicações existentes.

Atualmente, novas plataformas têm surgido com o intuito de melhorar a efetividade das discussões promovidas na Internet. Uma dessas plataformas é o Empurrando Juntos. 

% FALAR SOBRE O EMPURRANDO JUNTOS

Uma \textit{Application Programming Interface} (API), é uma interface que expõe os seus componentes como um serviço, permitindo que outras aplicações interajam com esses componentes \cite{wagh2012comparative, understanding_web}. Essa arquitetura de serviços possibilita o compartilhamento dos dados armazenados com as aplicações que consomem o serviço. 

O uso de uma API permite a convergência de informações e disseminação da sua proposta por meio de seu uso em diversos tipos de sistema.

% das denúncias e permite a criação de uma rede de apoio mais especializado por meio de questionários elaborados por estudiosos da temática.

% # FALAR SOBRE CLUSTERIZAÇÃO + empurrando juntos

\section{Objetivos}

% Considerando os aspectos supracitados, o objetivo do trabalho é criar uma API que consolide as informações das aplicações existentes e forneça informações para estudos que auxilie na descoberta de uma possível situação de violência e na criação de uma rede de apoio à mulher por meio de clusterização.

Considerando os aspectos supracitados, o objetivo do trabalho é criar uma API que auxilie na promoção de discussões acerca da violência contra a mulher utilizando como técnica a clusterização.

Para alcançar o objetivo proposto os objetivos específicos do trabalho são:
\begin{itemize}
	% \item Mapear as funcionalidades em comum entre as aplicações existentes;
	\item Realizar um levantamento sobre as aplicações existentes;
	% \item Definir as funcionalidades a serem consolidadas na API;
	\item Avaliar e selecionar algoritmos de clusterização;
	% \item Avaliar e selecionar algoritmos de clusterização;
\end{itemize}

\section{Etapas do trabalho}

Para realização do objetivo proposto serão realizadas cinco etapas: 
Diagnóstico, Definição do trabalho, Planejamento, Desenvolvimento, Aplicação em um caso de exemplo.

\noindent \textbf{Diagnóstico}: Esta etapa compreende o entendimento do cenário das aplicações implementadas no contexto de violência contra a mulher.
% , por meio do levantamento dos sistemas relacionados e mapeamento
% de funcionalidades em comum.

\noindent \textbf{Definição do trabalho}: Esta etapa tem como objetivo o estabelecimento do problema a ser solucionado e dos objetivos a serem alcançados.

\noindent \textbf{Planejamento}: Nesta etapa há o planejamento da realização do trabalho, como a definição de escopo e cronograma.

\noindent \textbf{Desenvolvimento}: Esta etapa compreende a realização de iterações para 
alcançar os objetivos definidos na etapa anterior.

\noindent \textbf{Aplicação em um caso de exemplo}: Esta etapa consiste na construção de um exemplo utilizando o
trabalho desenvolvido, a fim de ilustrar o seu uso.

% Levantar sistemas relacionados, Mapear funcionalidades em comum, Definir escopo da API,
% Modelar estrutura da API, Desenvolver a API e Desenvolver um caso de exemplo.

% Passos: Levantar sistemas relacionados, definir escopo da api, definir tecnologia, 
% definir um questionário padrão a ser respondido pelas mulheres para apoio a tomada de decisão, implementar api, definir categorias, adicionar inteligência, definir ações





% \section{Organização do trabalho}
