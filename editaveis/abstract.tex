\begin{resumo}[Abstract]
 \begin{otherlanguage*}{english}
  The growing number of discussions on political issues and other issues on social networks
  result in the polarization of those messages, considering the characteristics
  of the content selection algorithms used in these platforms. Therefore,
  the ``Cidade Democrática'' institute presents the idea of a new social participation
  platform, the ``Pushing Together'', to be used in web and mobile applications.
  The idea is to allow the user to create and participate of conversations, performing
  comments and/or votes on a comment of another participant. With the given votes, people
  who answered in a similar way are grouped, allowing the user to get a wide
  vision of all opinions.
  %
  The ``Pushing Together'' platform need to have these funcionalities of user/conversations management and user grouping to fulfill its purpose. Therefore, offering them as a web service would be meaningful contribution to the project. Moreover, a more flexible solution would be making possible to group the users using different configurable classification techniques.
  %
  An architecture was proposed and validated to meet the ``Pushing Together'' needs. This architecture consists in three independent modules, the Client module, the API module, and the Math module, which are responsible for the application graphic interface, the services that provides the presented functionalities, and the user grouping, respectively.
  %
  Considering the ``Pushing Together'' platform goals and needs,
  the goal of this study was the implementation of a RESTful API that holds all the functionalities mentioned above and that allows the use of different classification methods to group the users, working as a huge contribution to the platform.

   \vspace{\onelineskip}

   \noindent
   \textbf{Key-words}: Social participation. RESTful API. Pushing Together.
 \end{otherlanguage*}
\end{resumo}
