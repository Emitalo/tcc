\begin{resumo}[Abstract]
 \begin{otherlanguage*}{english}
  The growing number of discussions on political issues and other issues on social networks it turned out in a messages polarization considering the characteristics 
  of the content selection algorithms used in these platforms. In this sense, the Instituto Cidade Democrática presents the idea of a new social participation 
  platform, the ``Pushing Together'', to be used in web and mobile applications. The idea is allow the user to create and participate of conversations, realizing 
  comments and/or votes on a comment of another participant. With the votes, people who responded on similar way are grouped, allowing the user get a amplified 
  vision of all opinions. The grouping use the clustering method and it is the only module currently on development in the platform. In respect to the other popular 
  discussion issues it stands out the thematic of violence against women. This fact occurred due to increased the number of women victims of homicide and the search 
  for informations about that violence type. This scenario boosted the public policies creation for the reduction of this violence type and for the population 
  conscientize. Besides that, a data collect made by the authors shows that the applications and sites were created to help on the information divulgation and the 
  data and statistics collection. However, was not found a tool with the focus on effective discussions about this theme. Considering the ``Pushing Together'' need of 
  create a web applications and mobile applications, we define a service based architecture. In this scenario, the goal of this study was present a service 
  proposal, an API, for the ``Pushing Together'' platform, and evaluate it, if needed evolve it, to use in platforms that aims support women victims of violence.


   \vspace{\onelineskip}
 
   \noindent 
   \textbf{Key-words}: Social participation. API. Pushing Together. Violence against women.
 \end{otherlanguage*}
\end{resumo}
