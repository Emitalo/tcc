\begin{resumo}[Abstract]
 \begin{otherlanguage*}{english}
  The growing number of discussions on political issues and other issues on social networks
  result in the polarization of those messages, considering the characteristics 
  of the content selection algorithms used in these platforms. In this sense,
  the Cidade Democrática institute presents the idea of a new social participation 
  platform, the ``Pushing Together'', to be used in web and mobile applications.
  The idea is to allow the user to create and participate of conversations, performing 
  comments and/or votes on a comment of another participant. With the given votes, people 
  who answered in a similar way are grouped, allowing the user get a wide 
  vision of all opinions. The user grouping use the clustering method and it is the only
  module currently in development on the platform. In respect to other popular 
  discussion issues, the thematic of violence against women is one of them that stands out. This fact
  is due to the increasing number of women homicide victims and the search 
  for informations about that type of violence. This scenario boosted the public policies
  creation in order to reduce this type of violence and to increase population awareness.
  Besides that, a data collect made by the authors shows that
  applications and sites were created to help on the information divulgation and the 
  data and statistics collection. However, a tool with the focus on effective discussions about this theme was not found. 
  Considering the `` Pushing Together '' goals and needs, and also this gap identified on
  the plataforms of support to women victims of violence, 
  the goal of this study was present a service proposal, an API, for the `` Pushing Together ''
  platform and evaluate it, evolve it, if necessary, to use in platforms that aims to support women victims of violence.
  
  
   \vspace{\onelineskip}
 
   \noindent 
   \textbf{Key-words}: Social participation. API. Pushing Together. Violence against women.
 \end{otherlanguage*}
\end{resumo}
