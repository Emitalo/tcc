\begin{resumo}[Abstract]
 \begin{otherlanguage*}{english}
%   The growing number of discussions on political issues and other issues on social networks
%   result in the polarization of those messages, considering the characteristics 
%   of the content selection algorithms used in these platforms. Therefore,
%   the ``Cidade Democrática'' institute presents the idea of a new social participation 
%   platform, the ``Pushing Together'', to be used in web and mobile applications.
%   The idea is to allow the user to create and participate of conversations, performing 
%   comments and/or votes on a comment of another participant. With the given votes, people 
%   who answered in a similar way are grouped, allowing the user to get a wide 
%   vision of all opinions. The user grouping use the clustering method and it is the only
%   module currently in development on the platform.
%   Considering the `` Pushing Together '' goals and needs, 
%   the goal of this study is to present a service proposal, an API, for the `` Pushing Together ''
%   platform and evaluate it.
  
   \vspace{\onelineskip}
 
   \noindent 
   \textbf{Key-words}: Social participation. API. Pushing Together.
 \end{otherlanguage*}
\end{resumo}
