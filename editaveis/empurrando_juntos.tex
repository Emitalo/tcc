\chapter{Empurrando Juntos} \label{cap:empurrandojuntos}

O uso da tecnologia tem auxiliado no aumento da participação dos 
cidadãos no âmbito político e social de diversas maneiras. Discussões em sites, principalmente redes sociais, têm sido
destaques como uma forma recorrente desta participação \cite{marques2008participaccao}.
Contudo, considerando as características das plataformas utilizadas e a existência de grupos dominantes, interessados
na área passaram a observar um viés nas mensagens trocadas \cite{empurrandojuntos, marques2008participaccao}. 

O ``Empurrando Juntos'' surge nesse contexto, como uma plataforma capaz de contornar a situação de polarização nas discussões.
A ideia apresentada pelo Instituto Cidade Democrática tem como objetivo dar voz para a minoria e 
melhorar a efetividade dos debates e conversas de acordo com o seu propósito, possibilitando ao usuário
identificar grupos de opinião e tendências em uma conversa. Além disso, para promover a interação entre os membros agrupados em uma mesma classe 
e minimizar o fenômeno das bolhas de opinião, também há a proposta de gamificação da plataforma \cite{empurrandojuntos}.

A gamificação tem por objetivo utilizar mecanismos e elementos de jogos para prover o engajamento, motivação e desempenho
de uma pessoa na realização de uma tarefa \cite{pedreira2015gamification}. No caso do ``Empurrando Juntos'' a ideia é conceder poderes
especiais a determinados usuários de acordo com seus posicionamento nos grupos formados.

\section{Funcionamento do Empurrando Juntos}

  A Figura \ref{fig:resumo_ej} ilustra o funcionamento completo do sistema. O usuário entra no sistema e pode criar conversas ou
  comentar em conversas criadas por outros usuários. Para cada comentário realizado nessas conversas, é possível atribuir um 
  ``voto'' de concordância ou discordância do conteúdo exposto, ou não atribuir nenhum voto (pular).
  Os votos são entendidos como a opinião dos usuários e são utilizados para a formação dos grupos de pessoas de acordo com suas opiniões. 

  \begin{figure}[h!]
  \centering
  \includegraphics[scale=0.7]{figuras/resumo_ej.png}
  \caption{Funcionamento do ``Empurrando Juntos''}
  \label{fig:resumo_ej}
  \end{figure}

  Em relação a gamificação proposta, \citeonline{empurrandojuntos} criaram o conceito de ``the Push'' que consiste
  em prover notificações para determinados usuários e concedê-los o poder de enviar mensagens e criar eventos. Com 
  o intuito de movimentar ainda mais as participações na aplicação e criar comunicação entre os usuários agrupados.

  \citeonline{parra}, em um artigo para a página do Instituto Cidade Democrática, apresenta os três perfis de usuário
  que receberão essas notificações:

  \begin{enumerate}
    \item \textbf{Ativista de minoria}: usuários com representação em um grupo de minoria, ou seja, que não concordam com a opinião 
      expressa em maior volume;
    \item \textbf{Pessoa ponte de diálogo}: usuários com opiniões ``contraditórias'', que normalmente concordam com a maioria, mas
      possuem alguns pensamentos diferentes demonstrando compatibilidade com outros grupos;
    \item \textbf{Criador da consulta}: usuários específicos representantes de consultas de governos para auxiliar na criação de projetos
    e políticas.
  \end{enumerate}


  Contudo, a ideia da utilização dessas notificações e perfis ainda necessita de validações e deverá ser bem
  acompanhada para garantir a efetividade do seu propósito \cite{empurrandojuntos, parra}. 

  A Figura \ref{fig:prototipo} apresenta o protótipo definido para a plataforma e nela é possível ver a apresentação das conversas
  e grupos formados e os poderes concedidos ao usuário através do ``The Push'' \cite{parra}.

  \begin{figure}[h!]
  \centering
  \includegraphics[scale=0.3]{figuras/prototipo.png}
  \caption{Protótipo do ``Empurrando Juntos''. Fonte: \cite{parra}}
  \label{fig:prototipo}
  \end{figure}

  Levando isto em consideração, o escopo inicial do sistema foi definido para fornecer o gerenciamento de usuários, com cadastro, atualização, exclusão e autenticação, 
  incluindo contas provenientes do Facebook. Além disso, deve prover funcionalidades para o gerenciamento das conversas, comentários
  e votos. Por fim, deve prover também a funcionalidade principal de agrupamento dos usuários de acordo com os votos obtidos.
  
  Em suma, o objetivo da plataforma é realizar o agrupamento de pessoas que responderam de maneira parecida, ou seja, 
  concordaram e discordaram dos mesmos comentários. Com os grupos formados, a convergência e divergência 
  de opiniões e uma visualização mais efetiva da opinião das pessoas são apresentadas aos usuários.

\section{Classificação de dados no Empurrando Juntos}

  A funcionalidade de agrupamento de usuários pode ser implementada utilizando técnicas de classificação de dados.
  De acordo com a proposta descrita por \citeonline{empurrandojuntos}, 
  em um primeiro momento, essa classificação seria realizada utilizando algoritmos de clusterização.
  Uma breve explicação sobre classificação de dados pode ser encontrada no Apêndice \ref{apd:classificacao}.
  
  O algoritmo k-Means, um algoritmo baseado no centro, foi escolhido por \citeonline{tallys_tcc} 
  para ser o algoritmo utilizado no primeiro módulo de agrupamento de usuários do ``Empurrando Juntos''.
  
\section{Arquitetura atual do Empurrando Juntos}
  
  Atualmente, o ``Empurrando Juntos'' possui uma arquitetura monolítica acoplada a uma única implementação
  do módulo de agrupamento de usuários, não sendo possível utilizar outro algoritmo de clusterização para
  agrupar os usuários com base nos votos obtidos.
  
  \citeonline{tallys_tcc} criou uma primeira versão de um módulo de agrupamento de usuários
  para o ``Empurrando Juntos'' que implementava o algoritmo k-Means,
  o qual foi denominado de \textit{Pentano}, que apesar de ser uma alternativa \textit{open-source}
  ainda deixava o agrupamento de usuários acoplado a um único algoritmo de clusterização. 
  
  Neste trabalho propomos uma arquitetura modular e distribuída onde é possível
  trocar o módulo que realizará o agrupamento dos usuários.
  Para o ``Empurrando Juntos'' possuir um módulo de agrupamento de usuários, 
  nós refatoramos a aplicação \textit{Pentano} para se encaixar na arquitetura proposta.
  
  Como contribuição para o ``Empurrando Juntos'', o processo de clusterização e as
  demais funcionalidades supracitadas foram disponibilizados como serviços
  \textit{web} na API proposta, de forma que seja possível utilizar diferentes tipos de algoritmos para realizar a classificação dos dados.
  Para validação da API construída (Capítulo \ref{cap:aplicacao_exemplo}), uma implementação do algoritmo k-Means
  foi realizada para exemplificar o serviço de clusterização oferecido.


