\begin{resumo}
O crescente número de discussões acerca de temas políticos e outros temas nas redes sociais tem acarretado em uma polarização 
das mensagens trocadas devido às características dos algoritmos de seleção de conteúdo dessas plataformas. 
Nesse sentido, o Instituto Cidade Democrática apresenta a ideia de uma nova plataforma de participação social
que possa ser utilizada como aplicação Web e aplicativos, chamada de ``Empurrando Juntos''. 
O intuito é que o usuário crie e participe de conversas, realizando comentários e/ou votos em um comentário de outro participante. 
Com os votos realizados, as pessoas que responderem de maneira similar são agrupadas, 
provendo ao usuário uma visão ampliada das opiniões acerca do assunto. 
% 
Como o ``Empurrando Juntos'' possui a necessidade de ter essas funcionalidades de
gerenciamento de usuários, conversas e de agrupamento de usuários para cumprir o seu propósito,
oferecê-las como um serviço \textit{web} seria uma contribuição significante ao projeto.
Além disso, uma solução mais flexível seria possibilitar que  o agrupamento seja feito
utilizando diferentes técnicas de classificação configuráveis.
% 
Uma arquitetura foi proposta e validada para atender as necessidades do ``Empurrando Juntos'',
que consiste em três módulos independentes, o módulo cliente, o módulo de API e o módulo matemático,
os quais são responsáveis, respectivamente, pela interface gráfica da aplicação,
pelos serviços provedores das funcionalidades definidas e pelo agrupamento dos usuários.
% 
Considerando o objetivo e as necessidades da plataforma ``Empurrando Juntos'', 
o objetivo deste trabalho foi a implementação de uma API RESTful para o ``Empurrando Juntos''
que contemplasse as funcionalidades supracitadas e permitisse a utilização de diferentes
métodos de classificação para realizar o agrupamento dos usuários, servindo como contribuição para a plataforma.
\vspace{\onelineskip}
  
  
\noindent
\textbf{Palavras-chave}: Participação social. API RESTful. Empurrando Juntos.
\end{resumo}
