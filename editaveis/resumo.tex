\begin{resumo}
O crescente número de discussões acerca de temas políticos e outros temas nas redes sociais tem acarretado em uma polarização 
das mensagens trocadas devido às características dos algoritmos de seleção de conteúdo dessas plataformas. 
Nesse sentido, o Instituto Cidade Democrática apresenta a ideia de uma nova plataforma de participação social
que possa ser utilizada como aplicação Web e aplicativos, chamada de ``Empurrando Juntos''. 
O intuito é que o usuário crie e participe de conversas, realizando comentários e/ou votos em um comentário de outro participante. 
Com os votos realizados, as pessoas que responderem de maneira similar são agrupadas, 
provendo ao usuário uma visão ampliada das opiniões acerca do assunto. 
A arquitetura definida para a plataforma é modularizada, contendo três principais módulos: cliente, servidor e matemático.
Os quais são responsáveis, respectivamente, pela interface da aplicação, pelos serviços provedores das funcionalidades definidas 
e pelo agrupamento dos usuários. Além disso, o objetivo é que o agrupamento seja feito utilizando 
diferentes técnicas de classificação configuráveis.
Considerando o objetivo e as necessidades da plataforma ``Empurrando Juntos'', 
o objetivo deste trabalho foi a implementação de um serviço, uma API, 
para a plataforma ``Empurrando Juntos'' que contemplasse as funcionalidades supracitadas e permitisse a utilização de diferentes
módulos matemáticos. 
\vspace{\onelineskip}
  
\noindent
\textbf{Palavras-chave}: Participação social. API. Empurrando Juntos.
\end{resumo}
