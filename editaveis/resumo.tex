\begin{resumo}
O crescente número de discussões acerca de temas políticos e outros temas nas redes sociais tem acarretado em uma polarização 
dessas mensagens devido às características dos algoritmos de seleção do conteúdo a ser apresentado dessas plataformas. Nesse sentido, o Instituto Cidade Democrática 
apresenta a ideia de uma nova plataforma de participação social, o ``Empurrando Juntos'', que possa ser utilizada como aplicação Web e aplicativos. A ideia é que o 
usuário crie e participe de conversas, realizando comentários e/ou votos em um comentário de outro participante. Com os votos realizados, as pessoas que 
responderem de maneira similar são agrupadas, provendo ao usuário uma visão ampliada das opiniões acerca do assunto. O agrupamento utiliza a técnica de 
clusterização e é o único módulo atualmente em desenvolvimento na plataforma. No que se refere a outros tipos de discussões populares  destaca-se a temática de violência 
contra a mulher. Este fato decorre do aumento do número de mulheres vítimas de homicídio e da procura por informações sobre o assunto. Este cenário impulsionou a 
criação de políticas públicas para a redução desse tipo de violência e conscientização da população. Além disso, um levantamento realizado pelos autores 
mostra que aplicativos e sites surgiram para auxiliar na divulgação de informações e no levantamento de dados e estatísticas sobre esse tipo de violência. Contudo, nenhuma ferramenta com foco 
em discussões efetivas sobre o assunto foi encontrada. Considerando a necessidade da plataforma ``Empurrando Juntos'' de criar uma aplicação e aplicativos, 
definimos uma arquitetura baseada em serviços. E diante deste cenário, o objetivo deste trabalho foi apresentar uma proposta de um serviço, uma API, para a plataforma 
``Empurrando Juntos'' e avaliá-la, evoluí-la, se necessário, para uso em plataformas de apoio à mulheres vítimas de violência. 
\vspace{\onelineskip}
  
\noindent
\textbf{Palavras-chave}: Participação social. API. Empurrando Juntos. Violência contra a mulher.
\end{resumo}
