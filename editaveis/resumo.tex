\begin{resumo}
O crescente número de discussões acerca de temas políticos e outros temas nas redes sociais tem acarretado em uma polarização
das mensagens trocadas devido às características dos algoritmos de seleção de conteúdo dessas plataformas.
Nesse sentido, o Instituto Cidade Democrática apresenta a ideia de uma nova plataforma de participação social
que possa ser utilizada como aplicação Web e aplicativos, chamada de ``Empurrando Juntos''.
O intuito é que o usuário crie e participe de conversas, realizando comentários e/ou votos em um comentário de outro participante.
Com os votos realizados, as pessoas que responderem de maneira similar são agrupadas,
provendo ao usuário uma visão ampliada das opiniões acerca do assunto.
%

Como o ``Empurrando Juntos'' possui a necessidade de ter essas funcionalidades de
gerenciamento de usuários, conversas e de agrupamento de usuários para cumprir o seu propósito,
oferecê-las como um serviço \textit{web} seria uma contribuição significante ao projeto.
Além disso, uma solução mais flexível seria possibilitar que  o agrupamento seja feito
utilizando diferentes técnicas de classificação configuráveis.
%

O objetivo deste trabalho foi a implementação de uma API RESTful para o ``Empurrando Juntos''
que contemplasse as funcionalidades supracitadas e a proposta de uma arquitetura que permitisse a utilização de diferentes
métodos de classificação para realizar o agrupamento dos usuários.
%

O trabalho foi realizado em cinco etapas e a API foi implementada em seis iterações. Além do módulo de serviços (API), a arquitetura foi proposta com outros dois módulos, o módulo cliente, para prover a interface gráfica da plataforma, 
e o módulo matemático, responsável pelo agrupamento dos usuários. 
Ao final do desenvolvimento, foi construída uma aplicação para validação da API e da arquitetura proposta. Na validação apenas um módulo matemático foi integrado. A arquitetura proposta e a API foram consideradas adequadas e satisfatória para os requisitos identificados.
Contudo, foi percebida a necessidade de evoluções para outras funcionalidades do ``Empurrando Juntos'' essenciais
para a minimização da polarização das discussões e a carência de outros módulos matemáticos para teste da integração.
%

\vspace{\onelineskip}


\noindent
\textbf{Palavras-chave}: Participação social. API RESTful. Empurrando Juntos.
\end{resumo}
