\chapter{Proposta do trabalho} \label{cap:proposta}

Conforme descrito no Capítulo \ref{cap:introducao}, este trabalho será constituído de cinco etapas.
Em um primeiro momento, a etapa de diagnóstico foi realizada e os sistemas relacionados e suas funcionalidades em comum 
foram apresentadas no Capítulo \ref{cap:sistemas_relacionados}. Neste capítulo é apresentado o resultado da segunda etapa
que é a descrição da proposta do trabalho de acordo com os objetivos estabelecidos.

Para a plataforma ``Empurrando Juntos'', o lado servidor será implementado em uma aplicação chamada Pentano.
O Pentano será responsável por gerenciar as conversas, comentários, votos, agrupar os usuários
e cuidar de todos os aspectos da autenticação das aplicações.

O Pentano pode ser abstraído em dois módulos principais, que podem ser vistos na Figura \ref{fig:pentano}: o \textit{Server}, 
que é o módulo responsável por prover os serviços para as aplicações; e o \textit{Math} que é o módulo responsável por realizar
a clusterização, criando os grupos com os respectivos usuários.
Portanto, a contribuição deste trabalho será a criação do módulo \textit{Server} do Pentano, que proverá serviços para realizar as
propostas da plataforma ``Empurrando Juntos''.

\begin{figure}[h!]
\centering
\includegraphics[scale=0.8]{figuras/esquema_pentano.png}
\caption{Estrutura do Pentano}
\label{fig:pentano}
\end{figure}

\section{Requisitos da solução} \label{sec:requisitos}

Esta seção apresenta os requisitos funcionais e não funcionais identificados inicialmente para a aplicação.
Os requisitos aqui apresentados foram considerados, inicialmente, para a solidificação e validação da arquitetura proposta.

\subsection*{Requisitos funcionais}
  
    Uma avaliação do escopo da plataforma ``Empurrando Juntos'' permitiu o levantamento de características desejadas do sistema e,
    consequentemente, alguns requisitos associados, que foram sumarizados na Tabela \ref{tab:requisitos}.

    \begin{table}[h!]
    \centering
    \caption{Requisitos de alto nível da aplicação.}
    \label{tab:requisitos}
    \begin{tabular}{@{}cl@{}}
    \toprule
    \textbf{Característica}                      & \multicolumn{1}{c}{\textbf{Requisito}}                                                                                                \\ \midrule
    \multirow{3}{*}{Gerenciamento de Usuário}    & O sistema deve permitir o cadastro de usuários                                                                                        \\
						& O sistema deve permitir a alteração de usuários                                                                                       \\
						& O sistema deve permitir a exclusão de usuários                                                                                         \\
    \multirow{3}{*}{Gerenciamento de Conversa}   & O sistema deve permitir o cadastro de conversas                                                                                       \\
						& O sistema deve permitir a alteração de conversas                                                                                      \\
						& O sistema deve permitir a exclusão de conversas                                                                                       \\ \midrule
    \multirow{4}{*}{Gerenciamento de Comentário} & O sistema deve permitir o cadastro de comentários em conversas                                                                        \\
						& O sistema deve permitir a alteração de comentários                                                                                    \\
						& O sistema deve permitir a exclusão de comentários                                                                                     \\ 
						& O sistema deve permitir que o usuário vote em comentários                                                                             \\ \midrule
    Agrupamento de usuários                      & \begin{tabular}[c]{@{}l@{}}O sistema deve permitir a visualização dos usuários agrupados \\ de acordo com os votos dados\end{tabular} \\ \bottomrule
    \end{tabular}
    \end{table}


 \subsection*{Requisitos não funcionais}
    
      O Pentano deve prover serviços para realizar os requisitos funcionais listados acima.
      Estes serviços serão providos para a plataforma por meio de uma interface HTTP/HTTPS utilizando o estilo arquitetural REST.
      
      A autenticação e autorização das aplicações, e seus respectivos usuários, que utilizarão o Pentano deverá ser feita seguindo
      o protocolo OAuth \footnotemark, visando uma maior escalabilidade da solução futuramente.
      \footnotetext{OAuth Protocol. Disponível em \href{https://oauth.net/}{https://oauth.net}.}
      
      A tecnologia definida para implementação da solução proposta foi a linguagem Python juntamente com os \textit{frameworks}
      Django e Django Rest Framework, utilizando o banco de dados PostgreSQL.
      
      Outro aspecto importante para esta aplicação é a questão da manutenibilidade, pois como será um software livre, a solução
      deve ser desenvolvida de uma maneira que seja possível evoluir facilmente pela da comunidade.
      Com isto em mente, foram definidos seguintes padrões para o desenvolvimento da aplicação, considerando as tecnologias definidas:
      utilização do PEP8 \footnotemark como folha de estilos e a utilização da especificação JSON API \footnotemark para formatação das respostas da API.
      \footnotetext{Folha de estilos PEP8. Disponível em \href{https://www.python.org/dev/peps/pep-0008/}{https://www.python.org/dev/peps/pep-0008}.}
      \footnotetext{Especificação JSON API. Disponível em \href{http://jsonapi.org/}{http://jsonapi.org}.}
      
\section{Arquitetura da solução}

Para contemplar os requisitos elicitados foram mapeadas as entidades principais e os relacionamentos entre elas. 
Esse mapeamento pode ser visto na Figura \ref{fig:entidades}.

\begin{figure}[h!]
\centering
\includegraphics[scale=0.5]{figuras/entidades.png}
\caption{Entidades da API}
\label{fig:entidades}
\end{figure}

Considerando os requisitos funcionais e não funcionais da solução especificados anteriormente, a arquitetura da solução foi definida 
conforme a Figura \ref{fig:arquitetura_api}.

\begin{figure}[h!]
\centering
\includegraphics[scale=0.5]{figuras/arquitetura_api.png}
\caption{Entidades da API}
\label{fig:arquitetura_api}
\end{figure}

O módulo de serviço foi dividido em duas aplicações e ambas foram definidas em uma arquitetura
em 3 camadas: \textit{models}, \textit{views} e \textit{serializers}. 
A camada \textit{view} recebe e responde as requisições provenientes do cliente.
A camada \textit{serializer} é responsável pelo tratamento e formatação dos dados das \textit{models}
para renderização em JSON, seguindo a especificação JSON API.

Por fim, a camada \textit{model} contém aspectos negociais relacionados a cada uma das entidades definidas na 
Figura \ref{fig:entidades}. Além disso, nessa camada é estabelecida a comunicação com o módulo de clusterização através de sinais.
Os sinais criados são disparados durante as requisições de realização de novos votos e o módulo de clusterização é registrado
como \textit{listener} desse evento, reagrupando os usuários quando um novo voto é cadastrado.

Na Figura \ref{fig:resumo_ej_api} é apresentado o fluxo de funcionamento do ``Empurrando Juntos'' de acordo com a arquitetura
estabelecida para o Pentano.

\begin{figure}[bt!]
\centering
\includegraphics[scale=0.6]{figuras/resumo_ej_api.png}
\caption{Funcionamento do Empurrando Juntos - Comunicação entre os módulos}
\label{fig:resumo_ej_api}
\end{figure}

\vfill
\pagebreak

\section{Planejamento}

Para execução do trabalho foi realizado um planejamento que pode ser visto na Figura \ref{tab:planejamento}.
Foram contempladas as duas últimas etapas do trabalho, conforme definido no Capítulo \ref{cap:introducao}.
A etapa de Desenvolvimento da API será por meio de um processo iterativo e incremental com iterações de duas semanas. 

\begin{table}[h!]
\centering
\caption{Planejamento do trabalho}
\label{tab:planejamento}
\begin{tabular}{@{}cl@{}}
\toprule
\multicolumn{2}{c}{\textbf{Metas}}                                                                                                                         \\ \midrule
\multirow{3}{*}{Iteração 1}           & Estabelecimento da arquitetura                                                                                     \\
                                      & Implementação da feature de Gerenciamento de Usuário                                                               \\
                                      & Autenticação                                                                                                       \\ \midrule
\multirow{2}{*}{Iteração 2}           & Implementação da feature de Gerenciamento de Conversa                                                              \\
                                      & Implementação da feature de Gerenciamento de Comentário                                                            \\ \midrule
\multirow{2}{*}{Iteração 3}           & Implementação da feature de Agrupamento de usuários                                                                \\
                                      & Definição dinâmica da semântica dos votos                                                                          \\ \midrule
\multirow{2}{*}{Aplicação em um caso} & \begin{tabular}[c]{@{}l@{}}Estudo da adaptação da API para a temática de violência \\ contra a mulher\end{tabular} \\
                                      & Construção de um exemplo de uso da API                                                                             \\ \bottomrule
\end{tabular}
\end{table}

% \section{Resultados Esperados}








