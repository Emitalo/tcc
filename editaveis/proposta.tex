\chapter{Definição e planejamento da proposta} \label{cap:proposta}

 A seção \ref{requirements} apresenta os requisitos identificados para a solução.
 A seção \ref{architecture} apresenta a descrição da arquitetura definida para a solução.

\section{Requisitos da solução} \label{requirements}

    Considerando as necessidades do ``Empurrando Juntos'',
    foram identificados requisitos funcionais e não funcionais da aplicação,
    que são apresentados a seguir.

    \subsection*{Requisitos Funcionais} \label{functional_requirements}

    Uma avaliação do escopo da plataforma ``Empurrando Juntos'' permitiu o levantamento de características desejadas do sistema e,
    consequentemente, alguns requisitos associados, que foram sumarizados na Tabela \ref{tab:requisitos}.

    \begin{table}[h!]
    \centering
    \caption{Requisitos funcionais de alto nível da aplicação.}
    \label{tab:requisitos}
    \resizebox{\columnwidth}{!}{
      \begin{tabular}{@{}cl@{}}
      \toprule
      \textbf{Característica}                      & \multicolumn{1}{c}{\textbf{Requisito}}                                                                                                \\ \midrule
      \multirow{5}{*}{Gerenciamento de Usuário}    & O sistema deve permitir o cadastro de usuários                                                                                        \\
						  & O sistema deve permitir a alteração de usuários                                                                                       \\
						  & \begin{tabular}[c]{@{}l@{}}O sistema deve permitir a autenticação de usuários \\ cadastrados na plataforma\end{tabular}			\\
						  &  \begin{tabular}[c]{@{}l@{}}O sistema deve permitir a autenticação de usuários \\ cadastrados no Facebook\end{tabular}                     \\ \midrule
      \multirow{3}{*}{Gerenciamento de Conversa}   & O sistema deve permitir o cadastro de conversas                                                                                       \\
						  & O sistema deve permitir a alteração de conversas                                                                                      \\
						  & O sistema deve permitir a exclusão de conversas                                                                                       \\ \midrule
      \multirow{4}{*}{Gerenciamento de Comentário} & O sistema deve permitir o cadastro de comentários em conversas                                                                        \\
						  & O sistema deve permitir a alteração de comentários                                                                                    \\
						  & O sistema deve permitir a exclusão de comentários                                                                                     \\ 
						  & O sistema deve permitir que o usuário vote em comentários                                                                             \\ \midrule
      Agrupamento de usuários                      & \begin{tabular}[c]{@{}l@{}}O sistema deve permitir a 
						    visualização dos usuários agrupados \\ de acordo com os votos dados\end{tabular} \\ \bottomrule
	\end{tabular}
    }
	\end{table}
	  
      
    \subsection*{Requisitos não funcionais}	\label{non_functional_requirements}
    
    O ``Empurrando Juntos'' possui a característica de ser uma aplicação multiplataforma,
    que terá sua própria aplicação \textit{web} e aplicativos nas diversas plataformas existentes,
    o que cria uma necessidade das funcionalidades principais da aplicação serem reutilizáveis
    pelas diversas plataformas que possam existir.
    
    O agrupamento dos usuários geralmente é um processo computacionalmente custoso e recorrente. Por isso, o ``Empurrando Juntos''
    necessita de uma solução que realize esse processamento de forma assíncrona e distribuída, para que a comunicação com
    as aplicações clientes possam ser mais fluidas e que possa haver máquinas dedicadas ao processamento.
    
    Outro aspecto importante para esta aplicação é a questão da manutenibilidade, pois como é um software livre, a solução
    deve ser desenvolvida de uma maneira que seja possível evoluir facilmente pela comunidade.
    
    \subsection*{Decisões de implementação}
    
    Todos os serviços da API serão providos para a plataforma por meio de uma interface HTTP/HTTPS utilizando o estilo arquitetural REST.
    Para isso, a linguagem Python\footnote{Versão 3.4 - \href{https://www.python.org/download/releases/3.4.0/}{https://www.python.org/download/releases/3.4.0/}} 
    foi definida como tecnologia de implementação da solução proposta, juntamente com os \textit{frameworks}
    Django\footnote{Versão 1.11 - \href{https://www.djangoproject.com/}{https://www.djangoproject.com/}} e 
    Django Rest Framework\footnote{Versão 3.6.3 - \href{http://www.django-rest-framework.org/}{http://www.django-rest-framework.org/}}, 
    utilizando o banco de dados PostgreSQL\footnote{Versão 10.1 - \href{https://www.postgresql.org/}{https://www.postgresql.org/}}.
    A autenticação e autorização das aplicações e seus respectivos usuários serão realizadas por meio de JWT \textit{tokens}.
     
    Para que o agrupamento dos usuários possa ser realizado de forma assíncrona e distribuída, 
    foi utilizada uma plataforma de execução de tarefas assíncronas,
    o Celery\footnote{\href{http://www.celeryproject.org/}{http://www.celeryproject.org/}},
    que é uma aplicação de filas de tarefas assíncronas distribuídas focada em processamento em tempo real.
    O Celery foi escolhido por ser uma aplicação madura, estável, escrita em Python e bem acolhida pela comunidade Python,
    o que a torna de fácil integração com projetos nessa linguagem.
    Como a necessidade do ``Empurrando Juntos'' está mais ligada a operações em tempo real,
    processamento do agrupamento dos usuários à medida que os votos mudam,
    o Celery é adequado para fornecer as funcionalidades necessárias relacionadas ao processamento assíncrono e distribuído.
    
    Para executar as tarefas de forma assíncrona e distribuída, o Celery utiliza um \textit{message broker},
    que é uma plataforma intermediária entre duas aplicações para troca de mensagens,
    para enfileirar e delegar as tarefas para execução. 
    Para este trabalho, o RabbitMQ\footnote{\href{https://www.rabbitmq.com/}{https://www.rabbitmq.com/}}
    foi escolhido como \textit{message broker} por ser uma aplicação madura, tolerante a falhas e performática
    (escrito em Erlang \footnote{\href{https://www.erlang.org/}{https://www.erlang.org/}}),
    com armazenamento persistente das filas, suporte a \textit{backup}, suporte à \textit{plugins},
    possibilidade de formação de \textit{clusters} para alta disponibilidade,
    boa documentação e por ser \textit{open source}.
    Além disso, o RabbitMQ é o \textit{message broker} padrão do Celery, contando com uma boa integração com a ferramenta.
    
    Em relação à manutenibilidade da aplicação, considerando as tecnologias escolhidas,
    foram definidas os seguintes padrões para o desenvolvimento da aplicação:
    utilização do PEP8\footnote{Folha de estilos PEP8. Disponível em \href{https://www.python.org/dev/peps/pep-0008/}{https://www.python.org/dev/peps/pep-0008}.}
    como folha de estilos, por ser a folha de estilos oficial da linguagem Python,
    e a utilização da especificação JSON API\footnote{Especificação JSON API. Disponível em \href{http://jsonapi.org/}{http://jsonapi.org}.}
    para formatação das respostas da API, por ser uma especificação consolidada e bem documentada que possui padrões que
    se adequam às restrições do estilo REST.
    
\section{Planejamento do trabalho}

    Para a execução do trabalho foi utilizada uma abordagem ágil, onde os requisitos foram alocados em X iterações de duas semanas.
    
    [...]