\chapter{Clusterização} \label{cap:clusterizacao}

O processo de encontrar um modelo que possa descrever e distinguir classes de dados, ou seja, formar 
grupos, como proposto no projeto ``Empurrando Juntos'', é conhecido como classificação \cite{han2011data}.

\citeonline{tan2013data}, \citeonline{han2011data} e \citeonline{clustering_review} apresentam dois tipos de 
classificação: supervisionada e não supervisionada.
Na classificação supervisionada os objetos de dados são analisados a partir de rótulos de classe predefinidos nos objetos, ou seja, 
com as classes de dados previamente definidas e conhecidas. \citeonline{han2011data} chamam esse tipo de dado de ``dado treinado''
(dados cujos rótulos são conhecidos), onde o modelo que o descreve é obtido a partir da análise dos rótulos.
Na classificação não supervisionada, os rótulos de classe são obtidos a partir da análise dos dados dos objetos, tão somente \cite{tan2013data}.

Uma técnica de classificação não supervisionada é a clusterização \cite{clustering_review, tan2013data}, na qual os rótulos de classes dos objetos 
podem não existir ainda, como no caso do ``Empurrando Juntos''. O método empregado gera rótulos para um grupo de dados do conjunto \cite{han2011data}. 

A clusterização foi denominada por \citeonline{clustering_review} como uma técnica para organização ou agrupamento em \textit{clusters} 
de uma coleção de elementos que sigam padrões baseado em similaridade.
\citeonline{tan2013data} definem a análise de \textit{clusters}, ou clusterização, como a ação que agrupa objetos de dados
baseado apenas nas informações contidas nos dados do próprio objeto que permitam descrever os objetos e suas relações. Um \textit{cluster}
então pode ser visto como uma classe de objetos da qual é possível derivar regras para o grupo \cite{han2011data}.

Trata-se de uma técnica bastante utilizada em análise exploratórias, agrupamentos, 
tomada de decisão e implementações de aprendizado de máquina, como:
mineração de dados, recuperação de documentos, segmentação de imagens e padronização \cite{clustering_review}.

A técnica de clusterização pode ser dividida em alguns tipos: hierárquica ou particional; exclusiva, sobreposta ou \textit{fuzzy}; 
completa ou parcial \cite{tan2013data, clustering_review}. Na clusterização particional o agrupamento dos 
objetos de dados é realizado em simples \textit{clusters}, o que significa que pertencem apenas a um \textit{cluster} \cite{tan2013data}.

No processo de clusterização, os objetos de dados podem pertencer a um único \textit{cluster} exclusivamente ou podem
pertencer a diferentes \textit{clusters} ao mesmo tempo. O primeiro caso é conhecido como clusterização exclusiva e o segundo como sobreposta \cite{tan2013data}. 
Para todos os tipos de clusterização a serem realizados, alguns passos são necessários para obter os dados classificados ao final do processo.
% 
% \citeonline{clustering_review} definem uma clusterização hierárquica quando ela produz uma série
% de partições aninhadas baseadas em um critério para junção ou separação dos \textit{clusters} por meio de similaridade. 
% \citeonline{tan2013data} afirmam que esse tipo de clusterização acontece quando algoritmo
% produz um resultado no qual é possível obter \textit{subclusters} e dessa forma os \textit{clusters} são aninhados
% e organizados como uma árvore, seguindo a ideia de hierarquia.
% 
% Já o tipo \textit{fuzzy} trata da pertinência dos objetos a um grupo de forma probabilística, 
% ou em um grau de pertinência, em vez de determinar se o objeto pertence ou não àquele grupo \cite{tan2013data, clustering_review}.
% 
% Por fim, \citeonline{tan2013data} definem uma clusterização como completa quando no resultado todos os objetos estão em pelo menos um grupo. 
% Pois, na clusterização parcial alguns elementos podem ficar sem grupo caso não sejam compatíveis com algum \textit{cluster} formado.


\begin{figure}[h!]
\centering
\includegraphics[scale=0.5]{figuras/tasks_clustering.png}
\caption{Passos para clusterização. Adaptado de \citeonline{clustering_review}}
\label{fig:tasks_clustering}
\end{figure}

A clusterização pode ser resumida nos passos apresentados na Figura \ref{fig:tasks_clustering}.
A seleção de características é o processo de identificação do subconjunto das características para ser utilizado na clusterização, 
e a extração de características é o uso de uma ou mais transformações das características de entrada para 
torná-las mais efetivas para a clusterização. O objetivo de ambos é produzir o conjunto de objetos e dados a serem clusterizados \cite{clustering_review}.

A representação de padrões é o número de classes, número de padrões e o número, tipo e escala
das características disponíveis para a clusterização.
A similaridade é medida, em geral, por uma função de distância entre um par de objetos. Além disso, utiliza-se também
o conceito inverso de dissimilaridade para calcular o quanto objetos são diferentes. Essa função de distância pode variar de acordo 
com o contexto da aplicação e tipo dos dados \cite{clustering_review}. 

E por fim, o agrupamento pode ser realizado com uso de inúmeras técnicas e algoritmos diferentes \cite{clustering_review}.
Os objetos são agrupados (ou ``clusterizados'') buscando maximizar a similaridade intraclasse e minimizar a similaridade interclasse.
Em outras palavras, o objetivo é formar \textit{clusters} cujos elementos tenham alta similaridade entre si em um mesmo \textit{cluster} 
e baixa similaridade a elementos de outros \textit{clusters} \cite{han2011data}.

Apesar de parecer lógico e simples querer maximizar a similaridade intraclasse e minimizar a similaridade interclasse entre um conjunto de dados, 
\citeonline{tan2013data} exemplificam o quão difícil esta tarefa pode ser na Figura \ref{fig:clusters_difficulty}, onde temos um conjunto
de pontos (a) que pode ser agrupado de formas diferentes gerando quantidades diferentes de \textit{clusters} (b, c, d).

\begin{figure}[ht!]
\centering
\includegraphics[scale=0.4]{figuras/clusters_difficulty.png}
\caption{Formas de se agrupar o mesmo conjunto de dados. Fonte: \cite{tan2013data}}
\label{fig:clusters_difficulty}
\end{figure}

Por isso para a construção de um algoritmo de clusterização deve-se definir, além do algoritmo principal a ser utilizado, as funções
de cálculo de similaridade e dissimilaridade para delimitação dos \textit{clusters}.






