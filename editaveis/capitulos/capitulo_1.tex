\chapter{\textit{Application Programming Interface}}

De acordo com \citeonline{understanding_web}, uma API é uma interface que permite que os 
usuários interajam ou respondam a dados ou serviços solicitados de outro programa, outros
aplicativos ou sites. No contexto da Web, as APIs facilitam a troca de dados entre 
aplicações, permitem a criação de novas aplicações e constituem a base para o conceito de 
"Web como uma plataforma". 

\citeonline{wagh2012comparative} afirmam que existem vários métodos para interação entre o  
usuário e a Internet, através de serviços web (\textit{web services}), utilizando o conceito 
de arquitetura orientada a serviços (SOA). Essa arquitetura permite que os componentes de 
software, o que inclui funções, objetos e processos de diferentes sistemas, sejam expostos 
como um serviço.

Para prover esses serviços através de uma API, podem ser utilizadas duas arquiteturas: SOAP ou REST.

\section{\textit{Representational State Transfer} (REST) }

  A Transferência de Estado Representacional (do inglês, \textit{Representational State Transfer} ou REST)
  é um estilo arquitetural introduzido por \citeonline{fielding2002} que serve como um modelo abstrato
  para guiar o uso do \textit{Hypertext Transfer Protocol} (HTTP) e do \textit{Uniform Resource Identifier} (URI)
  na arquitetura da \textit{Web} moderna, abstraindo os elementos arquiteturais participantes de um sistema de
  hipermídia distribuído.
  
  \citeonline{perry1992} classificam os elementos arquiteturais em elementos de processamento (componentes),
  elementos de dados e elementos de conexão (conectores). O estilo REST foca no papel dos componentes, nas restrições
  da interação entre componentes e na interpretação de elementos de dados significantes para a interação \cite{fielding2002}.
  
  O estilo REST é um conjunto coordenado de restrições arquiteturais que possibilita baixo acoplamento entre os componentes,
  minimização da latência da comunicação e favorece a escalabilidade da implementação dos componentes \textit{fielding2002}.
  
  \subsection{Elementos de processamento}
  
    Componentes REST se comunicam transferindo uma representação do dado (informação) em algum formato padrão que pode
    ser selecionado dinamicamente com base nas preferências do destinatário \cite{fielding2002}. Além dos dados contidos na 
    informação, uma representação contém também metadados descrevendo os dados e, possivelmente, metadados sobre os
    metadados \cite{fielding2002}.
    
    O estilo REST possui uma natureza \textit{stateless} (sem estado), ou seja, cada requisição deve ser autosuficiente, sem
    dependência de outras interações. Isso possibilita a implementação de serviços utilizando uma estrutura complexa de componentes
    intermediários e múltiplos servidores de origem, caracterizando um aspecto interessante em um cenário distribuído \cite{fielding2002}.
    \citeonline{fielding2002} afirmam que a arquitetura REST foi desenvolvida para atender as necessidades de um sistema de
    hipermídia distribuído, onde a interação entre componentes consiste em transferência de grande quantidade de dados.
    
    A Tabela \ref{rest-components-elements} apresenta os tipos de componentes no REST.
  
    \begin{table}[ht!]
    \centering
    \caption{Tipos de componentes da arquitetura REST. Traduzido e adaptado de \cite{fielding2002}.}
    \label{rest-components-elements}
    \begin{tabular}{l|l}
    \hline
    \multicolumn{1}{c|}{\textbf{Tipos de componentes}} & \multicolumn{1}{c}{\textbf{Exemplos}} \\ \hline
    servidor de origem                                  & Apache httpd, Nginx                    \\ \hline
    \textit{gateway}                                    & Squid, CGI, Proxy reverso              \\ \hline
    proxy                                               & CERN Proxy, Netscape Proxy             \\ \hline
    agente de usuário                                   & Google Chrome, Mozilla Firefox, Lynx   \\ \hline
    \end{tabular}
    \end{table}
  
  \subsection{Elementos de dados}
    
    No estilo REST, uma informação é abstraída em um recurso, que pode ser qualquer informação que possa ser nomeada
    (documentos, imagens, serviços, entre outros).
    
    Para \citeonline{fielding2002}, um recurso é um mapeamento conceitual para um conjunto de entidades e não a entidade mapeada
    em um determinado espaço no tempo. Em termos matemáticos, um recurso \textit{R} é uma função de associação variável no tempo, a qual, em um 
    determinado tempo \textit{t}, mapeia um conjunto de valores, onde esses valores podem ser representações do recurso em questão ou
    outros identificadores de recursos.
    Com essa definição abstrata de um recurso é importante salientar que a semântica de um mapeamento de um recurso dever permanecer
    estática, uma vez que esta semântica é o que permite distinguir os recursos \cite{fielding2002}.
    Para identificar um recurso específico na interação entre dois componentes, o REST utiliza um identificador de recurso \cite{fielding2002}.
    
    A Tabela \ref{rest-data-elements} apresenta os elementos de dados no REST.
    
    \begin{table}[ht!]
    \centering
    \caption{Elementos de dados da arquitetura REST. Traduzido de \cite{fielding2002}.}
    \label{rest-data-elements}
    \begin{tabular}{l|l}
    \hline
    \multicolumn{1}{c|}{\textbf{Elementos de dados}} & \multicolumn{1}{c}{\textbf{Exemplos}}                                                             \\ \hline
    recurso                                           & \begin{tabular}[c]{@{}l@{}}alvo conceitual desejado de uma\\ referência de hipertexto\end{tabular} \\ \hline
    identificador de recurso                          & URI, URN                                                                                           \\ \hline
    representação                                     & documento HTML, imagem JPEG                                                                        \\ \hline
    metadados da representação                        & tipo de mídia, last-modified-time                                                                  \\ \hline
    metadados do recurso                              & link da fonte, alternativas                                                                        \\ \hline
    dados de controle                                 & if-modified-since, cache-control                                                                   \\ \hline
    \end{tabular}
    \end{table}
  
  \subsection{Elementos de conexão}
  
    Os conectores na arquitetura REST fornecem uma interface genérica para acessar e manipular o conjunto de valores de um ou mais recursos,
    encapsulando as atividades de acesso aos recursos e de transferência de representações de recursos \cite{fielding2002}.
    Os conectores gerenciam a comunicação de rede de um componente. 
    O encapsulamento possibilitado pelos conectores permite que seja possível se comunicar na \textit{Web} por meio de diferentes protocolos,
    embora o HTTP seja o protocolo de transferência primário \cite{fielding2002}.
    
    A Tabela \ref{rest-connectors-elements} apresenta os tipos de conectores no REST.
    
    \begin{table}[ht!]
    \centering
    \caption{Tipos de conectores da arquitetura REST. Traduzido de \cite{fielding2002}.}
    \label{rest-connectors-elements}
    \begin{tabular}{l|l}
    \hline
    \multicolumn{1}{c|}{\textbf{Tipos de conectores}} & \multicolumn{1}{c}{\textbf{Exemplos}} \\ \hline
    cliente                                            & libwww, libwww-perl                    \\ \hline
    servidor                                           & libwww, Apache API                     \\ \hline
    cache                                              & browser cache, Akamai cache network    \\ \hline
    resolvedor                                         & bind (biblioteca DNS)                  \\ \hline
    túnel                                              & SOCKS, SSL                             \\ \hline
    \end{tabular}
    \end{table}




