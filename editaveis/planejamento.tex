\chapter{Planejamento}

Conforme descrito no Capítulo \ref{cap:introducao}, este trabalho será constituído de cinco etapas.
Em um primeiro momento, a etapa de diagnóstico foi realizada e os sistemas relacionados e suas funcionalidades em comum foram apresentadas no Capítulo \ref{cap:sistemas_relacionados}.

Na Tabela \ref{tab:principais_funcionalidades} são apresentadas as principais funcionalidades em comum encontradas nos sistemas, de acordo com as categorias previamente estabelecidas.

\begin{table}[ht!]
\centering
\caption{Funcionalidades encontradas nos sistemas relacionados}
\label{tab:principais_funcionalidades}
\begin{tabular}{l|l}
\hline
\multicolumn{1}{c|}{\textbf{Categoria}} & \multicolumn{1}{c}{\textbf{Funcionalidades}} \\ \hline
\begin{tabular}[c]{@{}l@{}} Levantamento de Dados\\ e estatísticas\end{tabular} & \begin{tabular}[c]{@{}l@{}} - Categorização e junção dos dados obtidos de cada mulher \\ (através de questionário ou formulário de denúncia)\end{tabular} \\ \hline 
Mapeamento de Riscos & \begin{tabular}[c]{@{}l@{}}-  Mapeamento de locais de riscos por meio das denúncias de \\ violência realizadas\end{tabular} \\ \hline
Informativos & \begin{tabular}[c]{@{}l@{}} - Disponibilização de leis e conceitos relacionados \\ - Auxílio na descoberta da violência sofrida \\ - Disponibilização de notícias\end{tabular}  \\ \hline
Pedidos de Socorro & - Envio de pedido de socorro às autoridades competentes \\ \hline
Denúncia & \begin{tabular}[c]{@{}l@{}}- Denúncia da violência sofrida às autoridades competentes\\ - Compartilhamento da
violência sofrida para: \\ \hspace{1em} - alerta à outras mulheres \\ \hspace{1em} - levantamento de dados \end{tabular} \\ \hline
\end{tabular}
\end{table}


\section{Escopo da API}